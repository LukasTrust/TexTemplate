% \section{Results and Discussion}\label{Sec:Evaluation}
% 
% We evaluate the performance of our methods on the meshes of Fig.~\ref{fig:meshes}.
%     
% \begin{figure*}
%     \centering%
%     \def\svgwidth{\textwidth}%
%     \fontsize{6pt}{5pt}\selectfont%
%     \includesvg{figs/meshes}%
%     \caption[Test Meshes]{
%     $V$ denotes the number of vertices of the input mesh, $T$ the number of triangles and $M$ the number of meshlets coming from Meshoptimizer.
%     Meshlets are visualized with randomized colors.
%     }\label{fig:meshes}%
% \end{figure*}%
% 
% \begin{table}
%     \caption
%     [Comparison for the \textit{Rock} mesh.]
%     {    
%     We compare the optimal Gurobi and SCIP solutions against the sub-optimal stuff.
%     The CPU computation uses one thread per meshlet and was measured on an AMD Ryzen 9 7950X (16C/32T).    
%     }\label{tab:StripifyTable}    
%     {
%     \centering    
%     \input{tabs/stripify_table}
%     }
% \end{table}
% 
% 
% \begin{figure}
%     \includesvg[inkscapelatex=true]{figs/PDFTexFigure}
%     \caption[Vector graphics with fonts rendered by latex]{You can use Latex to render the fonts.}\label{fig:latexfonts}
% \end{figure}
% 
% \begin{figure}
%     \includesvg[inkscapelatex=false]{figs/SVGFigure}
%     \caption[Vector graphics with fonts directly taken from the file]{You can also use the fonts produced by the SVG file.}\label{fig:svgfonts}
% \end{figure}
% 
% \begin{figure}
%     \centering
%     \includegraphics[width=\columnwidth]{imgs/img.png}
%     \caption[Pixel graphics]{You can also directly include pixel graphics.}\label{fig:pixel}
% \end{figure}
% 
% Tab.~\ref{tab:StripifyTable} and Tab.~\ref{tab:ListOfSymbols} compare our optimal 
% method achieved with our other method
% of Sec.~\ref{Sec:MainPart} with Gurobi and SCIP against the sub-optimal strip of our 
% other implementation.
% See Figs.~\ref{fig:latexfonts}, Figs.~\ref{fig:svgfonts}, and Figs.~\ref{fig:pixel} for different types of figures.
% Here we test the acronyms.
% 
% % Test acronyms
% First time we use a \ac{GPU}.
% Next time, we use a \ac{GPU}.
% Now, we even have multiple \acp{GPU}.
