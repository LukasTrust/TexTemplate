Meshes are the standard representation for 3D geometry across a wide range of applications, 
including computer graphics, video games, CAD (Computer-Aided Design) and scientific vis
ualization. Usually constructed from triangle or quadrilateral elements, meshes provide control 
over surface shape, resolution and topology. These properties make meshes particularly valua
ble for applications that require precision in geometric representation. However, traditional ap
proaches to mesh processing such as mesh simplification, compression and surface reconstruc
tion often involve a trade-off between factors like accuracy, storage efficiency and processing 
speed. Achieving the right balance among these elements can be challenging, particularly when 
working with complex models or in real-time applications where computational resources are 
limited [Maglo+ 2015]. 
Recently, a new field of research has emerged that aims to use neural representations for en
coding 3D geometry, such as signed distance fields (SDFs), occupancy networks and neural 
implicit surfaces. These methods offer an alternative to traditional mesh representations by en
coding complex shapes in a compact and continuous format which is more flexible than tradi
tional meshes. By avoiding the rigid structure inherent in mesh-based approaches, neural rep
resentations enable smoother interpolations and differentiable processing, which can lead to 
more efficient and flexible methods for handling 3D data [Park+ 2019]. Despite these ad
vantages, one of the primary challenges remains, the conversion of these neural representations 
back into usable, topologically consistent meshes. This conversion process is computationally 
expensive, often lossy and typically fails to recover the fine details that would make these meth
ods truly viable for high-quality 3D applications [Sivaram+ 2024]. 
In response to these limitations, the concept of Neural Geometry Fields (NGFs) has been intro
duced as a potential solution. NGFs aim to combine the benefits of both neural networks and 
traditional meshes by creating a hybrid representation that has the advantage of the compactness 
of neural networks and the structural clarity of meshes. Instead of treating neural and mesh
based methods as separate or mutually exclusive, NGFs directly generate mesh geometry from 
neural features, thus bypassing many of the conversion challenges faced by implicit represen
tations. This approach has the potential to revolutionize how 3D geometry is processed, offering 
both the flexibility and precision necessary for high-fidelity models while reducing the compu
tational costs typically associated with traditional mesh-based techniques [Sivaram+ 2024].