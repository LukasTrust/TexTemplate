Meshes have long served as the standard for 3D geometry representation across numerous applications, such as computer graphics, CAD and scientific visualization, offering precise control over surface detail and topology.  
However, conventional mesh processing techniques, like simplification, compression and reconstruction, struggle with the trade-offs between accuracy, storage and computational efficiency.  
Recent innovations in neural representations, including signed distance fields and occupancy networks, present promising alternatives by encoding geometry in compact, continuous formats.  
Yet these methods face challenges in converting neural data back into topologically consistent meshes.  
Neural Geometry Fields (NGFs) emerge as a hybrid solution, blending the benefits of mesh structures with neural flexibility.  
This paper explores the fundamental principles of NGFs, evaluates their effectiveness in mesh generation, compression and surface reconstruction and assesses their potential for real-time high-fidelity 3D applications.  
We systematically compare NGFs with conventional techniques to highlight their strengths, limitations and transformative implications for 3D geometry processing.
